\section{Lecture 11: The Radix-2 Fast Fourier Transform}

Consider the DFT,
%
\begin{displaymath}
  X[k] = \sum_{n=0}^{N-1}x[n]\ex{-\im k\frac{2\pi}{N}n},\quad
  n,k = 0, 1, \hdots N-1 \,,
\end{displaymath}
%
which in abbreviated form becomes
%
\begin{displaymath}
  X[k] = \sum_{n=0}^{N-1}x[n]W_N^{kn} \,,
\end{displaymath}
%
with $W_N = \ex{-\im\frac{2\pi}{N}}$, and $W_N^m$ for $m=1,2,\hdots,N$
are the $n$\textsuperscript{th} roots of unity, which are evenly spaced
around the unit circle on the complex plane. Analysing the computational
complexity of this, to compute the $\{X[k]\}$, we need to compute $N^2$
complex multiplications and $N(N-1)$ complex additions (assuming the
$W_N^m$ are pre-calculated). Quadratic scaling in the problem complexity
is not tractable for large $N$, and consequently much work has been
conducted to find a more efficient way to compute the DFT.\\

The \textbf{Fast Fourier Transform} (FFT) is an algorithm which reduces
the number of operations to $\mathcal{O}(N\log N)$. The main tool we'll
use to achieve this is the decomposition of a large DFT into a series of
smaller DFTs. Furthermore, there are some simplifications we can apply
to the $W_N^{km}$ to simplify the calculation.

\subsection{Decimation in Time}
%
Consider $N$ as an even number. We can break the expression for the DFT
into separate sums over odd- and even-valued indices,
%
\begin{displaymath}
  X[k] = \sum_{n=0}^{N-1}x[n]W_N^{nk}
  = \sum_{n \mathrm{Even}}x[n] W_N^{nk}
  + \sum_{n \mathrm{Odd}}x[n] W_N^{nk} \,.
\end{displaymath}
%
Introducing a new index, $r$, such that $n=2r$,
%
\begin{align*}
  X[k] &= \sum_{r=0}^{N/2-1} x[2r] W_N^{2rk}
  + \sum_{r=0}^{N/2-1} x[2r+1] W_N^{(2r+1)k} \\
  &= \sum_{r=0}^{N/2-1}x[2r]\left(W_N^2\right)^{rk}
  + W_N^k\sum_{r=0}^{N/2-1}x[2r+1]\left(W_N^2\right)^{rk} \,,
\end{align*}
%
where each sum looks suspiciously like a DFT of length $N/2$. Indeed,
it transpires that each of these sums is precisely a length $N/2$ DFT,
since $W_N^2$ are the $N/2$ roots of unity. For example, consider
$W_4 = \{ \ex{-\im\frac{2\pi}{4}}, \ex{-\im\frac{4\pi}{4}}, \ex{-\im\frac{6\pi}{4}}, \ex{-\im\frac{8\pi}{4}} \}$,
the fourth roots of unity. Then,
$W_4^2 = \{ \ex{-\im\frac{4\pi}{4}}, \ex{-\im\frac{8\pi}{4}} \}$, the
second roots of unity, i.e. $W_2$. Consequently,
%
\begin{displaymath}
  X[k] = 
  \sum_{r=0}^{N/2-1}x[2r] W_{N/2}^{rk}
  + W_N^k\sum_{r=0}^{N/2-1}x[2r+1] W_{N/2}^{rk}
  = X_e[k] + W_N^k X_o[k] \,,
\end{displaymath}
%
where $X_e[k]$ is the DFT of the even-indexed elements of $x[n]$,
and $X_o[k]$ is the DFT of the odd-indexed elements of $x[n]$. The
computational process behind computing the $X[k]$ using this partition
into even and odd DFTs is shown in Figure \ref{fig::lecture_11_dft_one}.
%
\begin{figure}[H]
  \begin{tikzpicture}[
    thick, node distance=.5cm, circuit ee IEC,
    box/.style={
      draw, align=center, shape=rectangle, minimum width=1.5cm, minimum height=4cm,
      append after command={% see also: https://tex.stackexchange.com/a/129668
        \foreach \side in {east,west} {
          \foreach \i in {1,...,#1} {
            %          coordinate (\tikzlastnode-\i-\side)
            %          at ($(\tikzlastnode.north \side)!{(\i-.5)/(#1)}!(\tikzlastnode.south \side)$)
            (\tikzlastnode.north \side) edge[draw=none, line to]
            coordinate[pos=(\i-.5)/(#1)] (\tikzlastnode-\i-\side) (\tikzlastnode.south \side)
  }}}}]
  \node[box=4] (box-t) {$N/2$ \\ DFT};
  \node[box=4, below=of box-t] (box-b) {$N/2$ \\ DFT};

  \foreach \s[count=\i] in {0,2,4,6}
  \path (box-t-\i-west) edge node[at end, left]{$x[\s]$} ++(left:.5);
  \foreach \s[count=\i] in {1,3,5,7}
  \path (box-b-\i-west) edge node[at end, left]{$x[\s]$} ++(left:.5);

  \foreach \b/\s[count=\k] in {t/e, b/o}
  \foreach \i[evaluate={\j=int(\i-1)},
    evaluate={\J=int(ifthenelse(\k==2,\j+4,\j))}] in {1,...,4}
  \node [contact] (conn-\b-\i) at ([shift=(right:1.5)] box-\b-\i-east) {}
  edge node[above] {$X_{\s}[\j]$} (box-\b-\i-east)
  node [contact, label=right:{$X[\J]$}] (conn-\b-\i') at ([shift=(right:5)] box-\b-\i-east) {};

  \begin{scope}[every info/.append style={font=\scriptsize, inner sep=+.5pt}]

    \foreach \i[evaluate={\j=int(\i-1)},evaluate={\J=int(\i+3)}] in {1,...,4}
    \path (conn-t-\i) edge[current direction={pos=.27}] (conn-t-\i')
    edge[current direction={pos=.1}] (conn-b-\i')
    (conn-b-\i) edge[current direction={pos=.87, info={$W_N^\J$}}] (conn-b-\i')
    edge[current direction={pos=.9, info={$W_N^\j$}}] (conn-t-\i');
  \end{scope}
\end{tikzpicture}

  \caption{DFT for a length-8 signal. DFTs of length $N/2$ are taken for the
    even and odd-indexed elements of $x[n]$ and recombined to compute the full
    DFT of the original signal.
  }
  \label{fig::lecture_11_dft_one}
\end{figure}
%
We know that the number of complex multiplications in a DFT scales like $N^2$.
Splitting our calculation into two DFTs of length $N/2$ then turns the
complexity into $N^2 / 2$, i.e. we've halved the computational effort associated
with the complex multiplications. Note that we also have to multiply by the
$\{W_N\}$ to recover the $X[k]$, but this operation is linear in $N$ and so
doesn't affect the complexity. It should not have escaped the reader's attention
that we can repeat this trick again on the DFTs for $X_e[n]$ and $X_o[n]$.
We'll require the even and odd parts of $X_e[n]$ ($X_{ee}[n]$ and $X_{eo}[n]$,
respectively) and $X_o[n]$ ($X_{oe}[n]$ and $X_{oo}[n]$, respectively).
This scheme is depicted in Figure \ref{fig::lecture_11_dft_two}.
%
\begin{figure}[H]
  \begin{tikzpicture}[
  thick, node distance=.5cm, circuit ee IEC,
  box/.style={
    draw, align=center, shape=rectangle, minimum width=1.5cm, %minimum height=4cm,
    append after command={% see also: https://tex.stackexchange.com/a/129668
      \foreach \side in {east,west} {
        \foreach \i in {1,...,#1} {
          %          coordinate (\tikzlastnode-\i-\side)
          %          at ($(\tikzlastnode.north \side)!{(\i-.5)/(#1)}!(\tikzlastnode.south \side)$)
          (\tikzlastnode.north \side) edge[draw=none, line to]
          coordinate[pos=(\i-.5)/(#1)] (\tikzlastnode-\i-\side) (\tikzlastnode.south \side)
        }
      }
    }
  }
]


  \node[box=2, minimum height=2cm] (box-w) {$N/4$ \\ DFT};
  \node[box=2, below=of box-w, minimum height=2cm] (box-x) {$N/4$ \\ DFT};
  \node[box=2, below=of box-x, minimum height=2cm] (box-y) {$N/4$ \\ DFT};
  \node[box=2, below=of box-y, minimum height=2cm] (box-z) {$N/4$ \\ DFT};

  \foreach \s[count=\i] in {0,4}
  \path (box-w-\i-west) edge node[at end, left]{$x[\s]$} ++(left:.5);
  \foreach \s[count=\i] in {2,6}
  \path (box-x-\i-west) edge node[at end, left]{$x[\s]$} ++(left:.5);
  \foreach \s[count=\i] in {1,5}
  \path (box-y-\i-west) edge node[at end, left]{$x[\s]$} ++(left:.5);
  \foreach \s[count=\i] in {3,7}
  \path (box-z-\i-west) edge node[at end, left]{$x[\s]$} ++(left:.5);

  \node [contact] (conn-ee-0) at ([shift=(right:1.5)] box-w-1-east) {}
  edge node[above] {$X_{ee}[0]$} (box-w-1-east)
  node [contact, label=right:{$X_e[0]$}] (conn-e-0) at ([shift=(right:5)] box-w-1-east) {}
  node [contact, label=right:{$X[0]$}] (conn-0) at ([shift=(right:5)] conn-e-0) {};
  \node [contact] (conn-ee-1) at ([shift=(right:1.5)] box-w-2-east) {}
  edge node[above] {$X_{ee}[1]$} (box-w-2-east)
  node [contact, label=right:{$X_e[1]$}] (conn-e-1) at ([shift=(right:5)] box-w-2-east) {}
  node [contact, label=right:{$X[1]$}] (conn-1) at ([shift=(right:5)] conn-e-1) {};

  \node [contact] (conn-eo-0) at ([shift=(right:1.5)] box-x-1-east) {}
  edge node[above] {$X_{eo}[0]$} (box-x-1-east)
  node [contact, label=right:{$X_e[2]$}] (conn-e-2) at ([shift=(right:5)] box-x-1-east) {}
  node [contact, label=right:{$X[2]$}] (conn-2) at ([shift=(right:5)] conn-e-2) {};
  \node [contact] (conn-eo-1) at ([shift=(right:1.5)] box-x-2-east) {}
  edge node[above] {$X_{eo}[1]$} (box-x-2-east)
  node [contact, label=right:{$X_e[3]$}] (conn-e-3) at ([shift=(right:5)] box-x-2-east) {}
  node [contact, label=right:{$X[3]$}] (conn-3) at ([shift=(right:5)] conn-e-3) {};

  \node [contact] (conn-oe-0) at ([shift=(right:1.5)] box-y-1-east) {}
  edge node[above] {$X_{oe}[0]$} (box-y-1-east)
  node [contact, label=right:{$X_o[0]$}] (conn-o-0) at ([shift=(right:5)] box-y-1-east) {}
  node [contact, label=right:{$X[4]$}] (conn-4) at ([shift=(right:5)] conn-o-0) {};
  \node [contact] (conn-oe-1) at ([shift=(right:1.5)] box-y-2-east) {}
  edge node[above] {$X_{oe}[1]$} (box-y-2-east)
  node [contact, label=right:{$X_o[1]$}] (conn-o-1) at ([shift=(right:5)] box-y-2-east) {}
  node [contact, label=right:{$X[5]$}] (conn-5) at ([shift=(right:5)] conn-o-1) {};
  
  \node [contact] (conn-oo-0) at ([shift=(right:1.5)] box-z-1-east) {}
  edge node[above] {$X_{oo}[0]$} (box-z-1-east)
  node [contact, label=right:{$X_o[2]$}] (conn-o-2) at ([shift=(right:5)] box-z-1-east) {}
  node [contact, label=right:{$X[6]$}] (conn-6) at ([shift=(right:5)] conn-o-2) {};
  \node [contact] (conn-oo-1) at ([shift=(right:1.5)] box-z-2-east) {}
  edge node[above] {$X_{oo}[1]$} (box-z-2-east)
  node [contact, label=right:{$X_o[3]$}] (conn-o-3) at ([shift=(right:5)] box-z-2-east) {}
  node [contact, label=right:{$X[7]$}] (conn-7) at ([shift=(right:5)] conn-o-3) {};

  \begin{scope}[every info/.append style={font=\scriptsize, inner sep=+.5pt, below=3pt}]

    \path (conn-ee-0) edge[current direction={pos=.27}] (conn-e-0)
    edge[current direction={pos=.1}] (conn-e-2)
    (conn-eo-0) edge[current direction={pos=.9, info={$W_N^0$}}] (conn-e-0)
    edge[current direction={pos=.85, info={$W_N^4$}}] (conn-e-2);

    \path (conn-ee-1) edge[current direction={pos=.27}] (conn-e-1)
    edge[current direction={pos=.1}] (conn-e-3)
    (conn-eo-1) edge[current direction={pos=.9, info={$W_N^2$}}] (conn-e-1)
    edge[current direction={pos=.85, info={$W_N^6$}}] (conn-e-3);

    \path (conn-oe-0) edge[current direction={pos=.27}] (conn-o-0)
    edge[current direction={pos=.1}] (conn-o-2)
    (conn-oo-0) edge[current direction={pos=.9, info={$W_N^0$}}] (conn-o-0)
    edge[current direction={pos=.85, info={$W_N^4$}}] (conn-o-2);

    \path (conn-oe-1) edge[current direction={pos=.27}] (conn-o-1)
    edge[current direction={pos=.1}] (conn-o-3)
    (conn-oo-1) edge[current direction={pos=.9, info={$W_N^2$}}] (conn-o-1)
    edge[current direction={pos=.85, info={$W_N^6$}}] (conn-o-3);

    

    \path (conn-e-0) ++(1.25,0) edge[current direction={pos=.27}] (conn-0)
    edge[current direction={pos=.1}] (conn-4)
    (conn-o-0) ++(1.25,0) edge[current direction={pos=.95, info={$W_N^0$}}] (conn-0)
    edge[current direction={pos=.95, info={$W_N^4$}}] (conn-4);

    \path (conn-e-1) ++(1.25,0) edge[current direction={pos=.27}] (conn-1)
    edge[current direction={pos=.1}] (conn-5)
    (conn-o-1) ++(1.25,0) edge[current direction={pos=.95, info={$W_N^1$}}] (conn-1)
    edge[current direction={pos=.95, info={$W_N^5$}}] (conn-5);

    \path (conn-e-2) ++(1.25,0) edge[current direction={pos=.27}] (conn-2)
    edge[current direction={pos=.1}] (conn-6)
    (conn-o-2) ++(1.25,0) edge[current direction={pos=.95, info={$W_N^2$}}] (conn-2)
    edge[current direction={pos=.95, info={$W_N^6$}}] (conn-6);

    \path (conn-e-3) ++(1.25,0) edge[current direction={pos=.27}] (conn-3)
    edge[current direction={pos=.1}] (conn-7)
    (conn-o-3) ++(1.25,0) edge[current direction={pos=.95, info={$W_N^3$}}] (conn-3)
    edge[current direction={pos=.95, info={$W_N^7$}}] (conn-7);    
    
  \end{scope}

  
\end{tikzpicture}

  \caption{DFT for a length-8 signal using length-2 DFTs. DFTs of length $N/4$
    are taken for the even-even, even-odd, odd-even and odd-even indexed elements
    of $x[n]$ and recombined to compute the full DFT of the original signal.
  }
  \label{fig::lecture_11_dft_two}
\end{figure}
%
The computational complexity of this DFT is $N^2 / 4$ in the number of complex
multiplications, and so we've halved the computational effort again. In fact,
our efficiency is improved further when considering the form of a length-2 DFT:
%
\begin{displaymath}
  X[k] = \sum_{n=0}^1 x[n]W_2^{nk} = \sum_{n=0}^1 x[n](-1)^{nk} \,,
\end{displaymath}
%
since the second roots of unity are simply $1,-1$, and so
%
\begin{displaymath}
  X[0] = x[0] + x[1],\quad X[1] = x[0] - x[1] \,,
\end{displaymath}
%
which contains no complex multiplications whatsoever! The scheme becomes
like that depicted in Figure \ref{fig::lecture_11_dft_three}. The operation
of combining two values to yield two different values (the cross-shaped patterns
in the DFT figures) are referred to as \textbf{butterflies}.  
%
\begin{figure}[H]
  \begin{tikzpicture}[thick, node distance=.5cm, circuit ee IEC]

  \draw node (sig-0) at (0,0) {$x[0]$};
  \draw node[below= of sig-0] (sig-1) {$x[4]$};
  \draw node[below= of sig-1] (sig-2) {$x[2]$};
  \draw node[below= of sig-2] (sig-3) {$x[6]$};
  \draw node[below= of sig-3] (sig-4) {$x[1]$};
  \draw node[below= of sig-4] (sig-5) {$x[5]$};
  \draw node[below= of sig-5] (sig-6) {$x[3]$};
  \draw node[below= of sig-6] (sig-7) {$x[7]$};
  
  \node [contact, label=right:{$X_{ee}[0]$}] (box-w-1-east) at ([shift=(right:3)] sig-0) {};
  \node [contact, label=right:{$X_{ee}[1]$}] (box-w-2-east) at ([shift=(right:3)] sig-1) {};
  \node [contact, label=right:{$X_{eo}[0]$}] (box-x-1-east) at ([shift=(right:3)] sig-2) {};
  \node [contact, label=right:{$X_{eo}[1]$}] (box-x-2-east) at ([shift=(right:3)] sig-3) {};
  \node [contact, label=right:{$X_{oe}[0]$}] (box-y-1-east) at ([shift=(right:3)] sig-4) {};
  \node [contact, label=right:{$X_{oe}[1]$}] (box-y-2-east) at ([shift=(right:3)] sig-5) {};
  \node [contact, label=right:{$X_{oo}[0]$}] (box-z-1-east) at ([shift=(right:3)] sig-6) {};
  \node [contact, label=right:{$X_{oo}[1]$}] (box-z-2-east) at ([shift=(right:3)] sig-7) {};

  \begin{scope}[every info/.append style={font=\scriptsize, inner sep=+.5pt, below=5pt}]

    \path (sig-0) edge[current direction={pos=.27}] (box-w-1-east)
    edge[current direction={pos=.1}] (box-w-2-east)
    (sig-1) edge[current direction={pos=.9, info={$-1$}}] (box-w-2-east)
    edge[current direction={pos=.85, info={$1$}}] (box-w-1-east);

    \path (sig-2) edge[current direction={pos=.27}] (box-x-1-east)
    edge[current direction={pos=.1}] (box-x-2-east)
    (sig-3) edge[current direction={pos=.9, info={$-1$}}] (box-x-2-east)
    edge[current direction={pos=.85, info={$1$}}] (box-x-1-east);

    \path (sig-4) edge[current direction={pos=.27}] (box-y-1-east)
    edge[current direction={pos=.1}] (box-y-2-east)
    (sig-5) edge[current direction={pos=.9, info={$-1$}}] (box-y-2-east)
    edge[current direction={pos=.85, info={$1$}}] (box-y-1-east);

    \path (sig-6) edge[current direction={pos=.27}] (box-z-1-east)
    edge[current direction={pos=.1}] (box-z-2-east)
    (sig-7) edge[current direction={pos=.9, info={$-1$}}] (box-z-2-east)
    edge[current direction={pos=.85, info={$1$}}] (box-z-1-east);

  \end{scope}
  
  \node [contact] (conn-ee-0) at ([shift=(right:1.5)] box-w-1-east) {}
  node [contact, label=right:{$X_e[0]$}] (conn-e-0) at ([shift=(right:5)] box-w-1-east) {}
  node [contact, label=right:{$X[0]$}] (conn-0) at ([shift=(right:5)] conn-e-0) {};
  \node [contact] (conn-ee-1) at ([shift=(right:1.5)] box-w-2-east) {}
  node [contact, label=right:{$X_e[1]$}] (conn-e-1) at ([shift=(right:5)] box-w-2-east) {}
  node [contact, label=right:{$X[1]$}] (conn-1) at ([shift=(right:5)] conn-e-1) {};

  \node [contact] (conn-eo-0) at ([shift=(right:1.5)] box-x-1-east) {}
  node [contact, label=right:{$X_e[2]$}] (conn-e-2) at ([shift=(right:5)] box-x-1-east) {}
  node [contact, label=right:{$X[2]$}] (conn-2) at ([shift=(right:5)] conn-e-2) {};
  \node [contact] (conn-eo-1) at ([shift=(right:1.5)] box-x-2-east) {}
  node [contact, label=right:{$X_e[3]$}] (conn-e-3) at ([shift=(right:5)] box-x-2-east) {}
  node [contact, label=right:{$X[3]$}] (conn-3) at ([shift=(right:5)] conn-e-3) {};

  \node [contact] (conn-oe-0) at ([shift=(right:1.5)] box-y-1-east) {}
  node [contact, label=right:{$X_o[0]$}] (conn-o-0) at ([shift=(right:5)] box-y-1-east) {}
  node [contact, label=right:{$X[4]$}] (conn-4) at ([shift=(right:5)] conn-o-0) {};
  \node [contact] (conn-oe-1) at ([shift=(right:1.5)] box-y-2-east) {}
  node [contact, label=right:{$X_o[1]$}] (conn-o-1) at ([shift=(right:5)] box-y-2-east) {}
  node [contact, label=right:{$X[5]$}] (conn-5) at ([shift=(right:5)] conn-o-1) {};
  
  \node [contact] (conn-oo-0) at ([shift=(right:1.5)] box-z-1-east) {}
  node [contact, label=right:{$X_o[2]$}] (conn-o-2) at ([shift=(right:5)] box-z-1-east) {}
  node [contact, label=right:{$X[6]$}] (conn-6) at ([shift=(right:5)] conn-o-2) {};
  \node [contact] (conn-oo-1) at ([shift=(right:1.5)] box-z-2-east) {}
  node [contact, label=right:{$X_o[3]$}] (conn-o-3) at ([shift=(right:5)] box-z-2-east) {}
  node [contact, label=right:{$X[7]$}] (conn-7) at ([shift=(right:5)] conn-o-3) {};

  \begin{scope}[every info/.append style={font=\scriptsize, inner sep=+.5pt, below=3pt}]

    \path (conn-ee-0) edge[current direction={pos=.27}] (conn-e-0)
    edge[current direction={pos=.1}] (conn-e-2)
    (conn-eo-0) edge[current direction={pos=.9, info={$W_N^0$}}] (conn-e-0)
    edge[current direction={pos=.85, info={$W_N^4$}}] (conn-e-2);

    \path (conn-ee-1) edge[current direction={pos=.27}] (conn-e-1)
    edge[current direction={pos=.1}] (conn-e-3)
    (conn-eo-1) edge[current direction={pos=.9, info={$W_N^2$}}] (conn-e-1)
    edge[current direction={pos=.85, info={$W_N^6$}}] (conn-e-3);

    \path (conn-oe-0) edge[current direction={pos=.27}] (conn-o-0)
    edge[current direction={pos=.1}] (conn-o-2)
    (conn-oo-0) edge[current direction={pos=.9, info={$W_N^0$}}] (conn-o-0)
    edge[current direction={pos=.85, info={$W_N^4$}}] (conn-o-2);

    \path (conn-oe-1) edge[current direction={pos=.27}] (conn-o-1)
    edge[current direction={pos=.1}] (conn-o-3)
    (conn-oo-1) edge[current direction={pos=.9, info={$W_N^2$}}] (conn-o-1)
    edge[current direction={pos=.85, info={$W_N^6$}}] (conn-o-3);

    

    \path (conn-e-0) ++(1.25,0) edge[current direction={pos=.27}] (conn-0)
    edge[current direction={pos=.1}] (conn-4)
    (conn-o-0) ++(1.25,0) edge[current direction={pos=.95, info={$W_N^0$}}] (conn-0)
    edge[current direction={pos=.95, info={$W_N^4$}}] (conn-4);

    \path (conn-e-1) ++(1.25,0) edge[current direction={pos=.27}] (conn-1)
    edge[current direction={pos=.1}] (conn-5)
    (conn-o-1) ++(1.25,0) edge[current direction={pos=.95, info={$W_N^1$}}] (conn-1)
    edge[current direction={pos=.95, info={$W_N^5$}}] (conn-5);

    \path (conn-e-2) ++(1.25,0) edge[current direction={pos=.27}] (conn-2)
    edge[current direction={pos=.1}] (conn-6)
    (conn-o-2) ++(1.25,0) edge[current direction={pos=.95, info={$W_N^2$}}] (conn-2)
    edge[current direction={pos=.95, info={$W_N^6$}}] (conn-6);

    \path (conn-e-3) ++(1.25,0) edge[current direction={pos=.27}] (conn-3)
    edge[current direction={pos=.1}] (conn-7)
    (conn-o-3) ++(1.25,0) edge[current direction={pos=.95, info={$W_N^3$}}] (conn-3)
    edge[current direction={pos=.95, info={$W_N^7$}}] (conn-7);    
    
  \end{scope}

  
\end{tikzpicture}

  \caption{DFT for a length-8 signal using length-2 DFTs. DFTs of length $N/4$
    are taken for the even-even, even-odd, odd-even and odd-even indexed elements
    of $x[n]$ and recombined to compute the full DFT of the original signal.
  }
  \label{fig::lecture_11_dft_three}
\end{figure}
%
Our computational complexity can now be made more general. If $N$ is a power of 2,
we can recursively break the DFT into $\log_2(N)$ stages. At each stage, $N$ complex
multiplies are required for the multiplication by a $W_N^k$, meaning the scaling is
$N\log_2(N)$, as required for the FFT. This results in an enormous saving for large
$N$; consider $N=2^{10}$ -- quadratic scaling would require in excess of a million
complex multiplications, while the FFT requires roughly ten thousand, yielding a
speedup of $100\times$.\\

A further factor of two speedup can be obtained by noticing that the $W_N$
coefficients appearing in a butterfly are separated by a power of $N/2$. So
if one of the coefficients in the butterfly is $W_N^r$, then the corresponding
coefficient on the other arm of the butterfly will be $W_N^{r + N/2}$.
However,
%
\begin{displaymath}
  W_N^{r + \frac{N}{2}} = W_N^r W_N^{\frac{N}{2}} = - W_N^r \,,
\end{displaymath}
%
and only a single coefficient is actually used in each butterfly, scaled by
$\pm 1$. An efficiency can be realised by first multiplying the bottom input arm
of the butterfly by $W_N^r$, meaning the coefficient on the upper diagonal of
the butterfly is $1$ and that on the lower diagonal is $-1$. As such, $N$
complex multiplies at each stage of the FFT can be realised by $N/2$ complex
multiplies.\\

Since the $p\th$ and $q\th$ values in the $(m-1)\st$ stage are used to get the
$p\th$ and $q\th$ values in the $m\th$ stage, the computation can be done
in-place, i.e. so long as the input signal can be overwritten, the FFT can
be done without need for extra storage. The ordering of the input signal's
elements to compute the first stage's butterflies seems somewhat complex,
and it's not clear how one might generalise the ordering for a given $N$.
In the examples above for the length-8 FFT, for instance, the ordering is
$0,4,2,6,1,5,3,7$. However, if we write out the order of these indices
in binary and reverse them, it transpires that we recover the standard
progression $0,1,2,\hdots$,
%
\begin{align*}
  &000 \rightarrow 000 = 0, 001 \rightarrow 100 = 4, 010 \rightarrow 010 = 2,
  011 \rightarrow 110 = 6 \\
  &100 \rightarrow 001 = 1, 101 \rightarrow 101 = 5, 110 \rightarrow 011 = 3,
  111 \rightarrow 111 = 7 \,.
\end{align*}
%
So while the access pattern of these elements is non-trivial, the indices
to be extracted from the original signal can be obtained through a
\textbf{bit-reversal} algorithm.

\subsection{The Linear Algebraic View of the FFT}

Let's consider once again the length-8 DFT,
%
\begin{displaymath}
  \left[\begin{array}{c}
      X_0 \\ X_1 \\ X_2 \\ X_3 \\ X_4 \\ X_5 \\ X_6 \\ X_7
    \end{array}\right] =
  \left[\begin{array}{cccccccc}
      W_8^0 & W_8^0 & W_8^0 & W_8^0 & W_8^0 & W_8^0 & W_8^0 & W_8^0 \\
      W_8^0 & W_8^1 & W_8^2 & W_8^3 & W_8^4 & W_8^5 & W_8^6 & W_8^7 \\
      W_8^0 & W_8^2 & W_8^4 & W_8^6 & W_8^8 & W_8^2 & W_8^4 & W_8^6 \\
      W_8^0 & W_8^3 & W_8^6 & W_8^1 & W_8^4 & W_8^7 & W_8^2 & W_8^5 \\
      W_8^0 & W_8^4 & W_8^8 & W_8^4 & W_8^8 & W_8^4 & W_8^8 & W_8^4 \\
      W_8^0 & W_8^5 & W_8^2 & W_8^7 & W_8^4 & W_8^1 & W_8^6 & W_8^3 \\
      W_8^0 & W_8^6 & W_8^4 & W_8^2 & W_8^8 & W_8^6 & W_8^4 & W_8^2 \\
      W_8^0 & W_8^7 & W_8^6 & W_8^5 & W_8^4 & W_8^3 & W_8^2 & W_8^1      
    \end{array}\right]
  \left[\begin{array}{c}
      x_0 \\ x_1 \\ x_2 \\ x_3 \\ x_4 \\ x_5 \\ x_6 \\ x_7
    \end{array}\right] \,.
\end{displaymath}
%
The eighth roots of unity, however, can be simplified by noting that
$W_8^2 = -\im, W_8^4 = -1, W_8^6 = \im, W_8^8 = 1$. Furthermore, we
established in the previous section that $W_N^{r+\frac{N}{2}} = -W_N^r$.
Consequently, our DFT matrix exhibits some patterns,
%
\begin{displaymath}
  \left[\begin{array}{c}
      X_0 \\ X_1 \\ X_2 \\ X_3 \\ X_4 \\ X_5 \\ X_6 \\ X_7
    \end{array}\right] =
  \left[\begin{array}{cccccccc}
      1 & 1 & 1 & 1 & 1 & 1 & 1 & 1 \\
      1 & W_8^1 & -\im & W_8^3 & -1 & -W_8^1 & \im & -W_8^3 \\
      1 & -\im & -1 & \im & 1 & -\im & -1 & \im \\
      1 & W_8^3 & \im & W_8^1 & -1 & -W_8^3 & -\im & -W_8^1 \\
      1 & -1 & 1 & -1 & 1 & -1 & 1 & -1 \\
      1 & -W_8^1 & -\im & -W_8^3 & -1 & W_8^1 & \im & W_8^3 \\
      1 & \im & -1 & -\im & 1 & \im & -1 & -\im \\
      1 & -W_8^3 & \im & -W_8^1 & -1 & W_8^3 & -\im & W_8^1      
    \end{array}\right]
  \left[\begin{array}{c}
      x_0 \\ x_1 \\ x_2 \\ x_3 \\ x_4 \\ x_5 \\ x_6 \\ x_7
    \end{array}\right] \,.
\end{displaymath}
%
For the even-indexed columns, the upper 4 elements are equal to the
lower 4 elements. For odd-indexed columns, the upper 4 elements are
the negation of the lower 4 elements. We can eliminate this redundancy
by writing the even columns as follows:
%
\begin{displaymath}
  \mathrm{Even}(\matr{W}) =
  \left[\begin{array}{c}\matr{I}_4 \\ \matr{I}_4\end{array}\right]
  \left[\begin{array}{cccc}
      1 & 1 & 1 & 1 \\
      1 & -\im & -1 & \im \\
      1 & -1 & 1 & -1 \\
      1 & \im & -1 & -\im
    \end{array}\right] \,.
\end{displaymath}
%
But this sub-matrix of $\matr{W}$ is just the DFT matrix for a length-4 DFT.
We can similarly write the odd columns of $\matr{W}$ as
\begin{displaymath}
  \mathrm{Odd}(\matr{W}) =
  \left[\begin{array}{c}\matr{I}_4 \\ -\matr{I}_4\end{array}\right]
  \left[\begin{array}{cccc}
      1 & 1 & 1 & 1 \\
      W_8^1 & W_8^3 & -W_8^1 & -W_8^3 \\
      -\im & \im & -\im & \im \\
      W_8^3 & W_8^1 & -W_8^3 & -W_8^1
    \end{array}\right] =
  \left[\begin{array}{c}\matr{I}_4 \\ -\matr{I}_4\end{array}\right]
  \left[\begin{array}{cccc}
      1 & 0 & 0 & 0 \\
      0 & W_8^1 & 0 & 0 \\
      0 & 0 & W_8^2 & 0 \\
      0 & 0 & 0 & W_8^3
    \end{array}\right]
  \left[\begin{array}{cccc}
      1 & 1 & 1 & 1 \\
      1 & -\im & -1 & \im \\
      1 & -1 & 1 & -1 \\
      1 & \im & -1 & -\im
    \end{array}\right] \,,  
\end{displaymath}
%
where the length-4 DFT matrix now appears with some diagonal matrix
pre-multiplying it, which we'll call $\matr{T}_4$, the
\textbf{twiddle factors}. We can compose all of this together to obtain
%
\begin{displaymath}
  \vec{X} = \left[\begin{array}{cc}
      \matr{I}_4 & \matr{I}_4 \\
      \matr{I}_4 & -\matr{I}_4
    \end{array}\right]
  \left[\begin{array}{cc}
      \matr{I}_4 & \matr{0}_4 \\
      \matr{0}_4 & \matr{T}_4
    \end{array}\right] 
  \left[\begin{array}{cc}
      \matr{W}_4 & \matr{0}_4 \\
      \matr{0}_4 & \matr{W}_4
    \end{array}\right]
  \left[\begin{array}{c}
      \vec{x}_{\mathrm{even}} \\ \vec{x}_{\mathrm{odd}}
    \end{array}\right] \,.
\end{displaymath}
%
We now see what we've just derived for the FFT; the length-8 DFT can be
broken down into the length-4 DFTs of the even and odd elements of the
input signal, the odd DFT being modulated by some twiddle factors. Finally,
the first matrix appearing in this expression represents the sums and
differences of DFT values that were taken in the butterflies of the FFT.

\subsection{Decimation in Frequency}
%
Consider the DFT
%
\begin{displaymath}
  X[k] = \sum_{n=0}^{N-1}x[n]W_N^{nk} \,.
\end{displaymath}
%
The even-indexed frequencies, $X[2r]$, where $r=0,1,\hdots,\frac{N-1}{2}$,
are given by
%
\begin{align*}
  X[2r] &= \sum_{n=0}^{N-1}x[n]W_N^{2nr}
  = \sum_{n=0}^{N/2-1}x[n] W_N^{2nr} + \sum_{n=N/2}^{N-1}x[n]W_N^{2nr} \\
  &= \sum_{n=0}^{N/2-1}x[n] W_{N/2}^{nr} + \sum_{n=0}^{N/2-1}x[n+N/2]W_N^{2(n+N/2)r} \,,  
\end{align*}
%
where we have simply split the summation into two, and noticed that
%
\begin{displaymath}
  W_N^{2nr} =\ex{-\im\frac{4\pi n r}{N}} = \ex{-\im\frac{2\pi n r}{N/2}} = W_{N/2}^{nr} \,,
\end{displaymath}
%
and
%
\begin{displaymath}
  W_N^{2(n+N/2)r} = \ex{-\im\frac{4\pi n r}{N}}\ex{-\im\frac{2\pi N r}{N}}
  = \ex{-\im\frac{2\pi n r}{N/2}}\times 1 = W_{N/2}^{nr} \,.
\end{displaymath}
%
As such,
%
\begin{displaymath}
  X[2r] = \sum_{n=0}^{N/2-1}\left(x[n] + x[n+N/2]\right) W_{N/2}^{nr} \,.
\end{displaymath}
%
This is nothing more than a length-$\frac{N}{2}$ DFT whose input signal is
the original signal where elements separated by $\frac{N}{2}$ are summed. In
the same way, we can obtain odd $X[n]$,
%
\begin{displaymath}
  X[2r+1] = \sum_{n=0}^{N/2-1}\left(x[n] - x[n+N/2]\right) W_{N/2}^{nr} W_N^n \,,
\end{displaymath}
%
where we notice the addition of the twiddle factor $W_N^n$ and the fact that
the input signal is the difference between original elements in the input signal
separated by $\frac{N}{2}$. This strategy of decimation in frequency allows us to
obtain our $X[n]$ in a non-linear order as opposed to using the input signal $x[n]$
in a non-linear order. Decimation in frequency is completely equivalent to decimation
in time, but the implementer can select based on whether they have a preference for
the ordering in the time or frequency domain.
