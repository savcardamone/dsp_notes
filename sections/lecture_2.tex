\section{Lecture 2: Linear Time-Invariant Systems}

\subsection{Properties of Systems}
%
\begin{enumerate}
\item (\textbf{Causality})
  The output at time $n$ only depends on the input up to time $n$. For instance,
  $y[n] = x[n] - 2x[n-1]$ is causal, whereas $y[n] = x[n+3]$ is not causual.
  Real-time signals are necessarily causal, whereas signals used in image processing
  are not since the signal index denotes a pixel -- pixels ahead of the current
  pixel are accessible at an instance of time.

\item (\textbf{Linearity})
  For a system to be linear, it must exhibit \textbf{additivity} and
  \textbf{homogeneity}. Additivity requires that if $f(x_1) = y_1$ and
  $f(x_2) = y_2$, then $f(x_1 + x_2) = y_1 + y_2$. Homogeneity requires that
  $f(ax_1) = ay_1$ for some scalar $a$. Combining both of these properties, we can
  state that linearity is given by the requirement that
  %
  \begin{displaymath}
    f(ax_1[n] + bx_2[n]) = ay_1[n] + by_2[n] \,,
  \end{displaymath}
  %
  for any $x_1,x_2$.

\item (\textbf{Time Invariance})
  The system behaves the same way regardless of when the input is applied. So if
  $f(x[n]) = y[n]$, then $f(x[n-n_0]) = y[n-n_0]$.
  A useful way to verify time-invariance is through use of the delta function. We first evaluate
  the response to the input $x_1[n] = \delta[n]$, and then the response to the delayed input
  $x_2[n] = \delta[n+1]$. For the function in the example above, $y[n] = x[n^2]$, we see that
  the response to $x_1[n]$ yields
  %
  \begin{displaymath}
    y_1[0] = x_1[0] = 1,\quad y_1[1] = y_1[-1] = x_1[1] = 0,\quad y_1[2] = y_1[-2] = x_1[2] = 0,\quad \hdots \,,
  \end{displaymath}
  %
  and so one. However, the response to $x_2[n]$ yields
  %
  \begin{displaymath}
    y_2[0] = x_2[0] = 0,\quad y_2[1] = y_2[-1] = x_2[1] = 0,\quad y_2[2] = y_2[-2] = x_2[2] = 0,\quad \hdots \,,
  \end{displaymath}
  %
  which clearly isn't the same, and so the system is not time-invariant.
\end{enumerate}

\begin{exmp}
  Consider the function $y[n] = x[n] - 2x[n-1]$. Then, for $x_1$ and $x_2$, we have
  %
  \begin{align*}
    y_1[n] &= x_1[n] - 2x_1[n-1] \,, \\
    y_2[n] &= x_2[n] - 2x_2[n-1] \,.
  \end{align*}
  %
  The response to the input $z[n] = x_1[n] + x_2[n]$ is then
  %
  \begin{align*}
    y_z[n] = z[n] - 2z[n-1] &= (x_1[n] + x_2[n]) - 2(x_1[n-1] + x_2[n-1]) \\
    &= (x_1[n] - 2x_1[n-1]) + (x_2[n] - 2x_2[n-1) \\
    &= y_1[n] + y_2[n] \,.
  \end{align*}
  %
  Consequently, the system is additive.
\end{exmp}
%
\begin{exmp}
  Consider the same function as used in the previous example. The response to the
  input $z[n] = ax_1[n]$ is then
  %
  \begin{align*}
    y_z[n] = z[n] - 2z[n-1] &= ax_1[n] - 2ax_1[n-1] = a(x_1[n] - 2x[n-1]) \\
    &= ay_1[n] \,.
  \end{align*}
  %
  Consequently, the system is homogeneous.
\end{exmp}
%
\begin{exmp}
  Consider the function $y[n] = 3x[n] + 5$. The response to the input
  $z[n] = ax_1[n] + bx_2[n]$ is then
  %
  \begin{displaymath}
    y_z[n] = 3(ax_1[n] + bx_2[n]) + 5 \,.
  \end{displaymath}
  %
  However, $y_1[n] + y_2[n] = (3ax_1[n] + 5) + (3bx_2[n] + 5) = 3(ax_1[n] + bx_2[n]) + 10$,
  and so we fail to meet the additivity criterion. The system is then not linear.
\end{exmp}

  %
\begin{exmp}
  Consider the function $y[n] = x[n] - 2x[n-1]$. Then $y[n-n_0] = x[n-n_0] - 2x[n-n_0-1]$.
  We create a delayed signal $z[n] = x[n-n_0]$, such that
  %
  \begin{displaymath}
    y_z[n] = x[n-n_0] - 2x[n-n_0-1] \,,
  \end{displaymath}
  %
  which is equal to $y[n-n_0]$, and so the system is time-invariant.
\end{exmp}
%
\begin{exmp}
  Consider the function $y[n] = x[n^2]$. Then
  $y[n-n_0] = x[(n-n_0)^2] = x[n^2-2nn_0 + n_0^2]$. As before, we create a delayed signal
  $z[n] = x[n-n_0]$, such that
  %
  \begin{displaymath}
    y_z[n] = z[n^2] = x[n^2 - n_0] \,.
  \end{displaymath}
  %
  This is clearly not the same as $y[n-n_0]$, and so the system is not time-invariant.
\end{exmp}
  
\subsection{Linear Time-Invariant (LTI) Systems}
%
Real-world systems are often modelled as LTI systems since they're often a very good approximation,
and their analysis is very easy. The key concept that facilitates their treatment is that of
superposition, such that
%
\begin{equation}
  a_0x[n] + a_1x[n-1] + a_2x[n-2] + \hdots = a_0y[n] + a_1y[n-1] + a_2y[n-2] + \hdots \,.
\end{equation}
%
This is important because we strive to use the delta function, and constructing signals as
sums of shifted delta functions. The above is simply the expanded form of the sifting property
of the delta function,
%
\begin{displaymath}
  x[n] = \sum_{k=-\infty}^\infty \delta[n-k]x[k] \,,
\end{displaymath}
%
such that the response of an LTI system, $H$, to a signal $x[n]$ is given by
%
\begin{equation}
  H(x[n]) = H\left( \sum_{k=-\infty}^\infty \delta[n-k]x[k] \right) \,.
\end{equation}
%
However, the $x[k]$ are effectively constant amplitudes that scale the delta functions,
so we can invoke the linearity condition,
%
\begin{equation}
  H(x[n]) = \sum_{k=-\infty}^\infty x[k] H(\delta[n-k]) \,.
\end{equation}
%
Finally, by time-invariance,
%
\begin{equation}
  H(x[n]) = \sum_{k=-\infty}^\infty x[k] h[n-k] \,.
\end{equation}
%
where $h[n-k]$ is referred to as the \textbf{unit impulse response}, and dictates how
the system behaves when presented with the delta function $\delta[n-k]$. So for
instance if the system has $\delta[0]$ as an input, the output will likely persist
for some duration owing to the system's inner machinations ``smearing'' the input out
in time, and the unit impulse response captures this.
The summation is a special mathematical form, referred to as convolution and denoted by
%
\begin{equation}
  y[n] = x[n] * h[n] \,.
\end{equation}
%
The unit impulse response fully characterises an LTI system.
%
\begin{exmp}
  Consider the impulse response $\delta[n-N]$ convolved with some generic input,
  %
  \begin{displaymath}
    y[n] = x[n] * \delta[n-N] = \infsum{k}x[n]\delta[n-k-N] = x[n-N] \,,
  \end{displaymath}
  %
  i.e. the unit impulse response $\delta[n-N]$ has the effect of delaying the
  signal by $N$ units.
\end{exmp}

\subsection{Importance of Phasors in LTI Systems}
%
Consider the causal convolutional sum
%
\begin{displaymath}
  y[n] = x[n] * h[n] = \sum_{k=0}^{N-1} h[k]x[n-k] \,, 
\end{displaymath}
%
a sum of delayed inputs $x[n-k]$ that are weighted by $h[k]$. Define
the input as a complex sinusoid, $x[n] = \ex{\im\omega n}$. Then
%
\begin{align*}
  y[n] &= \sum_{k=0}^{N-1} h[k]x[n-k] = \sum_{k=0}^{N-1} h[k]\ex{\im\omega(n-k)}
  = \sum_{k=0}^{N-1} h[k]\ex{\im\omega n}\ex{-\im\omega k} \\
  &= \ex{\im\omega n}\sum_{k=0}^{N-1}h[k]\ex{-\im\omega k}
  = x[n] \sum_{k=0}^{N-1}h[k]\ex{-\im\omega k} \,.
\end{align*}
%
An LTI system acting on a complex sinusoid then involves a calculation involving
only phasors, with phasor magnitude $h[k]$ and phasor angle $-\omega k$. Since
all signals can be expressed as a linear combination of
complex sinusoids, this treatment can be applied to any signal by expanding it
into a weighted sum of complex sinusoids -- this is the motivation for
the Fourier transform, which we'll encounter in a later lecture.
