\section{De/Modulation}
%
Virtually the entirety of the electromagnetic spectrum can be manipulated so as
to carry data. Practically, this means that we have some information carrying
signal, that we'll call the \textbf{modulation signal}, $s_m(t)$, that we want
to be placed at/around some designated frequency that we'll call the
\textbf{carrier signal}, $s_c(t)$. Then, the carrier is effectively impressed
with the data-carrying signal; this process is referred to as
\textbf{modulation}, and the opposite process (the extraction of the modulation
signal from the carrier that has been impressed with the modulation signal) is
referred to as \textbf{demodulation}.\\
%
There are several ways in which this de/modulation can be accomplished, with a
few being in the common vernacular -- AM and FM, for instance, designate
modulation schemes that are used in broadcasting. We'll attempt to provide an
overview of a number of commonly used de/modulation schemes in this section.

\subsection{Amplitude Modulation (AM)}
%
Amplitude Modulation is perhaps the simplest modulation scheme to consider.
Let the modulation and carrier signals be defined by
%
\begin{displaymath}
  s_m(t) = \Re\left\{ A \exp\left[ \im\omega_m t \right] \right\}
  \qquad\mathrm{and}\qquad
  s_c(t) = \Re\left\{ \exp\left[ \im\omega_c t \right] \right\} \,,
\end{displaymath}
%
where $\omega_c \gg \omega_m$ (think of radio broadcasting where one tunes into
the carrier at 90MHz, but audible frequencies are in the range of 20Hz to
20kHz). Then, upon mixing these two signals, 
%
\begin{align*}
  s(t) &= s_m(t) s_c(t)
  = \frac{A}{4}\left( \ex{\im\omega_m t} + \ex{-\im\omega_m t}\right)
  \left(\ex{\im\omega_c t} + \ex{-\im\omega_c t}\right) \\
  &= \frac{A}{4} \left(
    {\red{\ex{\im\left(\omega_c + \omega_m\right) t}}} +
    {\blu{\ex{\im\left(\omega_c - \omega_m\right) t}}} +
    {\blu{\ex{-\im\left(\omega_c - \omega_m\right) t}}} +
    {\red{\ex{-\im\left(\omega_c + \omega_m\right) t}}}
  \right)\\
  &= \frac{A}{2} \left(
    {\red{\cos\left(\left(\omega_c + \omega_m\right) t\right)}} +
    {\blu{\cos\left(\left(\omega_c - \omega_m\right) t\right)}}
  \right) \,,
\end{align*}
%
we obtain a superposition of signals with frequencies $\omega_c \pm
\omega_m$.\\

Of course, this frequency mixing generalises to more than single
frequency modulation signals. As we've previously mentioned, audible frequencies
span the range $[20,20\times 10 ^3]$Hz, and so to broadcast audio signals,
$s_m(t)$ has frequency content across this range.\\
%
Consider some baseband signal with arbitrary frequency content up to
$\pm\omega_m$, whose spectrum is designated $S_m(\omega)$. Then
since mixing in the time domain is convolution in the frequency domain, and our
carrier signal is a delta $S_c(\omega) = \delta(\omega \pm \omega_c)$, the
resultant signal is $S(\omega) = S_m(\omega) \ast \delta(\omega \pm \omega_c)$
which is symmetric about $\omega_c$ with bandwidth $2\omega_m$; the regions
above and below the carrier are referred to as \textbf{sidebands}, and the
transmission scheme we've just described is
\textbf{double-sideband transmission}.\\
%
Demodulation of this signal is easily achieved by mixing $s(t)$ with a carrier
on the receiver side (sometimes referred to as a \textbf{local oscillator}. We
see that this recovers $s_m(t)$ as follows:
%
\begin{align*}
  s(t) s_c(t) &= \left(
    {\red{\cos\left(\left(\omega_c + \omega_m\right) t\right)}} +
    {\blu{\cos\left(\left(\omega_c - \omega_m\right) t\right)}}
  \right) {\grn{\cos\left(\omega_c t\right)}}\\
  &= \cancelto{0}{\cos\left(\left( 2\omega_c \pm \omega_m \right)t\right)} +
    \cos\left(\pm\omega_m t\right) = s_m(t) \,,
\end{align*}
%
where the first term cancels to zero since the signal is low-pass filtered after
mixing with the carrier. One can avoid mixing with the carrier on the receiver
side by implementing an amplitude envelope detector, since $s(t)$ is nothing
more than the carrier whose amplitude envelope is modulated by $s_m(t)$.\\

Clearly there are some inefficiencies in this scheme. Mixing with the carrier
results in a peaked spectrum of $S(\omega)$ at $\omega_c$, which carries no
useful information about the modulated signal. One can suppress this frequency
in $S(\omega)$ through notch filtering, resulting in higher efficiency at the
transmitter, referred to as
\textbf{double-sideband suppressed-carrier} (DSB-SC) transmission, although this
necessitates a more complex receiver since the carrier must be regenerated prior
to retrieval of the modulated signal.\\
%
Another inefficiency is that $S(\omega)$ is symmetric about $\omega_c$, and
consequently each sideband contains the same information. The signal then has a
bandwidth of $2\omega_m$, even though only half of this contains useful
information. One can address this waste of bandwidth through filtering one of
the sidebands to yield \textbf{single-sideband suppressed-carrier} (SSB-SC)
transmission, but this once again necessitates a more complex receiver since the
filtered sideband must be regenerated to retrieve $s_m(t)$.

\subsection{Frequency Modulation (FM)}
%

\subsection{Phase-Shift Keying (PSK)}
%

\subsection{Quadrature Amplitude Modulation (QAM)}
%
As the name suggests, QAM is something of an extended version of AM. The basic
premise is that two carriers are used that are in-quadrature ($\pi/2$ radians
out of phase) with one another, each being modulated with an independent
baseband signals, $I(t)$ and $Q(t)$, referred to as the in-phase and quadrature
components.\\
%
Let's consider the baseband signals as a single complex baseband signal,
%
\begin{displaymath}
  g(t) = I(t) + \im Q(t) \,,
\end{displaymath}
%
and map onto a complex carrier, $s_c(t) = \ex{\im\omega_c t}$, such that the two
carriers are just a cosine and sine, and so by definition are in-quadrature with
one another. Then, mixing these two signals together,
%
\begin{align*}
  s(t) &= g(t) s_c(t) = \left[I(t) + \im Q(t)\right]\ex{\im\omega_c t} =
    \left[I(t) + \im Q(t)\right]\left[ \cos \omega_c t + \im\sin\omega_c t\right] \\
  &= {\red{I(t)\cos\omega_c t - Q(t)\sin\omega_c t}} + \im\left[
    {\blu{I(t)\sin\omega_c t + Q(t)\cos\omega_c t}}
  \right] \,.
\end{align*}
%
Of course, even though we're working with complex signals, when dealing with
``real-world'' quantities that get transmitted and received, we have to take the
real part of the expression, meaning that only the red portion of the above is
ever manipulated. As a result, the transmitter implementation is
straightforward: we have parallel inputs $I(t)$ and $Q(t)$ that are mixed
independently with the in-quadrature carriers $\cos\omega_c t$ and
$-\sin\omega_ct$, respectively. These two signals are then superimposed and
transmitted. We've already seen what happens to the individual mixed signals in
the frequency domain in the section on amplitude modulation, and since the
Fourier transform is linear, the spectrum of the superposition of signals is
just the sum of the individual spectra.\\

To demodulate the signal, we must mix the received signal $s(t)$ with the
conjugate carrier, $\ex{-\im\omega_c t}$ on the receiver end:
%
\begin{align*}
  \Re\left\{s(t)\right\} \ex{-\im\omega_c t} &= \left[
    I(t)\cos\omega_c t - Q(t)\sin\omega_c t
  \right] \left[ \cos\omega_c t - \im\sin\omega_c t\right] \\
  &= I(t) \left[\cos^2\omega_c t -\im\cos\omega_c t \sin\omega_c t\right] -
    Q(t)\left[ \sin\omega_c t\cos\omega_c t -\im\sin^2\omega_c t\right] \\
  &= \frac{I(t)}{2}\left[
    1 + \cos\left(2\omega_c t\right) - \im\sin\left(2\omega_c t\right)
  \right] + \frac{\im Q(t)}{2}\left[
    1 - \cos\left(2\omega_c t\right) + \im\sin\left(2\omega_c t\right) 
  \right] \\
  &= \frac{I(t) + \im Q(t)}{2} +
    \cancelto{0}{\mathcal{O}\left( 2\omega_c \right)} \\
  &= \frac{g(t)}{2}\,,
\end{align*}
%
where we've assumed the mixed result is low-pass filtered to remove the terms
that have frequency $2\omega_c$. The result is then the complex baseband signal,
whose individual terms are recovered by mixing either of the in-quadrature
carriers (remember that in reality there's no such thing as a complex carrier --
we either mix the cosine or the sine component, recovering $I(t)$ or $Q(t)$,
respectively).
