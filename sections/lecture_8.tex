\section{Lecture 8: The Z-Transform}

We have established that the continuous-time Fourier transform
%
\begin{displaymath}
  X(\omega) = \int\dx{t}x(t)\ex{-\im\omega t}
\end{displaymath}
%
has a discrete-time counterpart
%
\begin{displaymath}
  X(\omega) = \infsum{n}x[n]\ex{-\im\omega n} \,.
\end{displaymath}
%
The Fourier transform is actually a specialised example of
the Laplace transform,
%
\begin{equation}
  X(s) = \int\dx{t}x(t)\ex{-st}
\end{equation}
%
where $s\in\mathbb{C}$. While the Fourier transform is a complex
function of a real variable $\omega$, the Laplace transform is a
complex function of a complex variable, $s$. The discrete-time
version of the Laplace transform is the \textbf{Z-transform}, is
the Laurent series
%
\begin{equation}
  X(z) = \infsum{n}x[n]z^{-n} \,,
\end{equation}
%
for complex $z$. For discrete-time LTI systems, $z^n$ is ``special''.
Consider the case where $x[n] = z^n$. Then
%
\begin{align*}
  y[n] &= \infsum{k}h[k]x[n-k] = \infsum{k} h[k]z^{n-k} = z^n\infsum{k}h[k]z^{-k} \\
  &= H(z)z^n \,,
\end{align*}
%
where $H(z)$ is the Z-transform of $h[k]$, and is referred to as the
\textbf{transfer function}. Notice that this is similar to the
frequency response: $y[n] = H(z)z^n$ modulates the input $z^n$ by some
complex-valued $H(z)$.

\subsection{Z-Transform and DTFT}
%
Looking back at our definitions of the DTFT and Z-transform, it is
clear that
%
\begin{displaymath}
  X(\omega) = \infsum{n}x[n]\ex{-\im\omega n} = \left.\vphantom{\frac{1}{1}}X(z)\right|_{z=\ex{\im\omega}} \,.
\end{displaymath}
%
Since $|\ex{-\im\omega}| = 1$ and is $2\pi$-periodic, the DTFT
samples those $z$ that lie on the unit circle in the complex plane.
The Laplace transform places no such restriction on the location of
$z$ and is consequently more general. This increased generality is useful
since the DTFT doesn't always converge/ exist, meaning we have another
tool to fall back on. Recall that a sufficient condition for
the existence of the DTFT is the absolute summability of the signal, but
not all signals satisfy this constraint, and consequently can't be
treated with the DTFT. As such,
%
\begin{enumerate}
\item The Z-transform may converge in places where the DTFT doesn't exist.
\item The notation is generally simpler; the Z-transform is typically
  characterised by polynomials or rational functions in $z$.
\item The Z-transform helps when designing filters.
\end{enumerate}

\subsection{Convergence of the Z-Transform}
%
By writing $z$ in polar form $z = r\ex{\im\omega}$,
%
\begin{displaymath}
  X(z) = X\left(r\ex{\im\omega}\right) = \infsum{n}x[n]z^{-n}
  = \infsum{n}\left(x[n]r^{-n}\right)\ex{-\im\omega n} \,,
\end{displaymath}
%
which is a DTFT of the signal $x[n]r^{-n}$. Consequently, we can say
that $X(z)$ converges when $x[n]r^{-n}$ is absolutely summable.
%
\begin{exmp}
  For the step function,
  $\lim{N\rightarrow\infty}\sum_{n=-\infty}^N|u[n]|\rightarrow\infty$,
  and as a result isn't amenable to DTFT methods. However,
  %
  \begin{displaymath}
    \infsum{n}u[n]r^{-n} = \sum_{n=0}^\infty r^{-n} = \frac{1}{1-r^{-n}}
  \end{displaymath}
  %
  when $|r| > 1$. The \textbf{region of convergence} (ROC) of the
  Z-transform of $u[n]$ is $|r| > 1$, i.e. everything outside of the
  unit circle in the complex plane. We see that points on the unit
  circle are outside of the ROC, hence the failure of the DTFT.
\end{exmp}
%
Note that convergence of the Z-transform depends only on the magnitude of
$z$, and consequently ROCs are annular regions on the complex plane. If
the ROC contains the unit circle, then the DTFT exists. If the Z-transform
of an impulse response for an LTI system converges on the unit circle,
then the system is said to be \textbf{stable}. We will see that for
much of the time we can write the Z-transform as the quotient of
polynomials in $z$,
%
\begin{displaymath}
  X(z) = \frac{N(z)}{D(z)} \,.
\end{displaymath}
%
When $N(z)\rightarrow 0$, $X(z)\rightarrow 0$. The roots of this
polynomial are referred to as \textbf{zeroes}. When $D(z)\rightarrow 0$,
$X(z)\rightarrow\infty$. The roots of this polynomial are referred
to as \textbf{poles}. The Z-transform requires the definiton of \textit{both} the
ROC and the algebraic form for $X(z)$ so that the poles and zeroes can
be ascertained.
%
\begin{exmp}
  Consider the right-sided exponential, $x[n] = a^n u[n]$. When $a<1$, we
  have a decreasing exponential, which is absolutely summable. However, when
  $a>1$, we have an increasing exponential, which is not. By using the
  Z-transform,
  %
  \begin{displaymath}
    X(z) = \infsum{n}x[n]z^{-n} = \sum_{n=0}^\infty a^n z^{-n}
    = \sum_{n=0}^\infty \left(\frac{a}{z}\right)^n \,,
  \end{displaymath}
  %
  which converges when $\left|\frac{a}{z}\right| < 1$, i.e. $|z| > |a|$.
  Assuming this convergence criteria are met,
  %
  \begin{displaymath}
    X(z) = \frac{1}{1 - \frac{a}{z}} = \frac{z}{z-a} \,.
  \end{displaymath}
  %
  This transfer function has a zero at $z=0$ and pole at $z=a$. We can
  depict all of this in a \textbf{Pole-Zero Plot}, as per Figure QQ.
\end{exmp}
%
\begin{exmp}
  Consider the left-sided exponential, $x[n] = -a^n u[-n-1]$, which like
  the right-sided exponential, is not absolutely summable for $a>1$. Then,
  %
  \begin{displaymath}
    X(z) = -\infsum_{n}a^n u[-n-1]z^{-n}
    = -\sum_{n=-\infty}^{-1}\left(\frac{a}{z}\right)^n
    = -\sum_{n=1}^{\infty}\left(\frac{z}{a}\right)^n \,.
  \end{displaymath}
  %
  This converges if $\left|\frac{z}{a}\right| < 1$, i.e. when $|z| < |a|$.
  Invoking the geometric series, we have
  %
  \begin{displaymath}
    X(z) = -\frac{1}{1 - \frac{z}{a}} + 1 = -\frac{a}{a - z} + \frac{a-z}{a - z}
    = \frac{z}{z-a} \,,
  \end{displaymath}
  %
  where we add $1$ to account for the summation excluding the $n=0$ term. This
  has the same poles and zeroes as the right-sided exponential. The difference is
  that the region of convergence now lies within the circle of radius $a$.
\end{exmp}
%
\begin{exmp}
  Consider the signals $x_1[n] = \left(\frac{1}{2}\right)^nu[n]$ and
  $x_2[n] = \left(-\frac{1}{3}\right)^n u[n]$. Denote the composite signal
  $x[n] = x_1[n] + x_2[n]$. We can treat these terms separately:
  %
  \begin{displaymath}
    X_1(z) = \frac{z}{z - \frac{1}{2}}, \qquad X_2(z) = \frac{z}{z + \frac{1}{3}} \,.
  \end{displaymath}
  %
  We find the ROC for $X_1(z)$ is $|z|>\frac{1}{2}$ and that for $X_2(z)$ is
  $|z| > \frac{1}{3}$. Consequently, the ROC for $X(z)$ is $|z| > \frac{1}{2}$,
  the intersection of both ROCs. To find the poles and zeroes for the composite
  function,
  %
  \begin{displaymath}
    X(z) = \frac{z}{z - \frac{1}{2}} + \frac{z}{z + \frac{1}{3}}
    = \frac{2z^2 - \frac{1}{6}z}{(z-\frac{1}{2})(z+\frac{1}{3})}
    = \frac{2z(z-\frac{1}{12})}{(z-\frac{1}{2})(z+\frac{1}{3})} \,.
  \end{displaymath}
  %
  As such, there are zeroes at $|z|=0,\frac{1}{12}$ and poles at
  $|z|=\frac{1}{2},\frac{1}{3}$.
\end{exmp}
%
\begin{exmp}
  Consider the signal
  %
  \begin{displaymath}
    x[n] = \left\{\begin{array}{ccl}
    a^n & & 0 \leq n < N \\
    0 & & \mathrm{otherwise}
    \end{array}\right. \,,
  \end{displaymath}
  %
  and associated Z-Transform
  %
  \begin{displaymath}
    X(z) = \sum_{n=0}^{N-1}a^n z^{-n} = \sum_{n=0}^{N-1}\left(\frac{a}{z}\right)^n \,.
  \end{displaymath}
  %
  Using the finite sum formula $\sum_{k=0}^{n-1}ar^{-k} = a\left(\frac{1-r^n}{1 - r}\right)$,
  %
  \begin{displaymath}
    X(z) = \frac{1 - \left(\frac{a}{z}\right)^N}{1 - \frac{a}{z}} = \frac{(z-a)^N}{z^N - az^{N-1}}
    = \frac{(z-a)^N}{z^{N-1}(z-a)} \,.
  \end{displaymath}
  %
  Both numerator and denominator are $N$\textsuperscript{th} order polynomials in
  $z$, and from the fundamental theorem of algebra there are $N$ roots in
  both the numerator and denominator, i.e. $N$ zeroes and $N$ poles. From the
  denominator, we'll have $N-1$ poles at $|z|=0$ and a final pole at $|z|=|a|$.
  From the numerator, we'll have $N$ zeroes at the complex roots of $z^n = a^n$,
  i.e. $|a|\ex{2\pi \im n / N}$ for $0\leq n < N$. However, note that all of
  these roots at $|z| = |a|$ are effectively undefined since they arise at
  $X(z) = \frac{0}{0}$. Consequently, the ROC is simply $|z| > 0$. This is a
  common situation for inputs of finite length -- nothing can really go wrong.
  The poles start moving and becoming problematic for inputs that are
  infinitely long.
\end{exmp}
%
\begin{exmp}
  Consider the signal $x[n] = 2^n\cos(3n)u[n]$. Our first thought should be
  to turn any sinusoids into exponentials,
  %
  \begin{displaymath}
    x[n] = 2^{n-1}\left(\ex{\im 3n} + \ex{-\im 3n}\right) = \frac{1}{2}\left[
      \left(2\ex{3\im}\right)^n + \left(2\ex{-3\im}\right)^n 
    \right] u[n] \,.
  \end{displaymath}
  %
  In this form, we can compute the transfer function through summing
  the transfer functions for right-sided exponentials, which we've previously
  evaluated
  %
  \begin{displaymath}
    X(z) = \frac{1}{2}\left[\frac{z}{z - 2\ex{3\im}} + \frac{z}{z - 2\ex{-3\im}}\right]
    = \frac{1}{2}\left[\frac{z(z-2\ex{-3\im}) + z(z - 2\ex{3\im})}{(z-2\ex{-3\im})(z-2\ex{3\im})}\right] \,.
  \end{displaymath}
  %
  From this, we can establish the existence of two poles at $2\ex{\pm3\im}$.
  Doing some more manipulations,
  %
  \begin{align*}
    X(z) &= \frac{
      2z^2 - 2z\left(\ex{-3\im} + \ex{3\im}\right)
    }{
      2\left(z^2 - 2z\left(\ex{-3\im} + \ex{3\im}\right)\right) + 4
    }
    = \frac{2z^2 - 4\cos(3)}{2z^2 - 8z\cos(3) + 8}
    = \frac{z^2 - 2z\cos(3)}{z^2 - 4z\cos(3) + 4} \\
    &= \frac{z(z - 2\cos(3))}{z^2 - 4z\cos(3) + 4} \,,
  \end{align*}
  %
  and we have zeroes at $|z| = 0, 2\cos(3)$. We saw that the ROC for a right
  sided exponential was when $|z| > |a|$, which in our case equates to
  $|a| = \left|2\ex{\pm 3\im}\right|$ (since this is where the poles are),
  and consequently we have $|z| > 2$.
\end{exmp}
%
\begin{exmp}
  Consider the signal $x[n] = r^n\sin(\omega_0 n)u[n]$. Our first thought should be
  to turn any sinusoids into exponentials,
  %
  \begin{displaymath}
    x[n] = \frac{r^n}{2}\left(\ex{\im \omega_0n} + \ex{-\im \omega_0n}\right) = \frac{1}{2}\left[
      \left(r\ex{\im\omega_0}\right)^n + \left(r\ex{-\im\omega_0}\right)^n 
    \right] u[n] \,.
  \end{displaymath}
  %
  In this form, we can compute the transfer function through summing
  the transfer functions for right-sided exponentials, which we've previously
  evaluated
  %
  \begin{displaymath}
    X(z) = \frac{1}{2}\left[\frac{z}{z - r\ex{\im\omega_0}} - \frac{z}{z - r\ex{-\im\omega_0}}\right]
    = \frac{1}{2}\left[\frac{z(z-r\ex{-\im\omega_0}) - z(z - r\ex{\im\omega_0})}{(z-r\ex{-\im\omega_0})(z-r\ex{\im\omega_0})}\right] \,.
  \end{displaymath}
  %
  From this, we can establish the existence of two poles at $r\ex{\pm\im\omega_0}$.
  Doing some more manipulations,
  %
  \begin{align*}
    X(z) &= \frac{
      z^2 + zr(\ex{\im\omega_0} - \ex{-\im\omega_0}) - z^2)
    }{
      2(z^2 - zr(\ex{\im\omega_0} + \ex{-\im\omega_0} + r^2)
    }
    = \frac{zr\sin(\omega_0)}{z^2 - 2zr\cos(\omega_0) + r^2} \\
    &=  \frac{z^{-1}r\sin(\omega_0)}{1 - 2z^{-1}r\cos(\omega_0) + z^{-2}r^2}\,,
  \end{align*}
  %
  and we have zeroes at $|z|$. We saw that the ROC for a right
  sided exponential was when $|z| > |a|$, which in our case equates to
  $|a| = \left|r\ex{\pm \im\omega_0}\right|$ (since this is where the poles are),
  and consequently we have $|z| > |r|$.
\end{exmp}

\subsection{Rules About the ROC}
%
\begin{enumerate}
\item The ROC is a rang or disk centred at the origin.
\item The ROC contains no poles.
\item If $x[n]$ is of finite length, then the ROC is the complex
  plane with the exception of $0,\infty$.
\item If $x[n]$ is left-sided, then the ROC is contained within a cicle.
  If $x[n]$ is right-sided, then the ROC is outside of a circle. If $x[n]$
  is two-sided then the ROC is the annular region bounded by poles.
\item The ROC for a transfer function $H(z)$ tells us whether the system
  is stable and causal. If the system is stable, the ROC includes the unit
  circle, $|z| = 1$. If the system is causal, the ROC must contain infinity and
  the impulse response will be right-sided. The best case scenario is all poles
  contained within the unit circle and the ROC extending to infinity, and the
  system is consequently stable and causal.
\end{enumerate}
