\section{Decibels}
%
When dealing with the power content of a signal, we are typically presented with enormous
dynamic ranges spanning many orders of magnitude; picowatts and below all the way to
gigawatts and above. Relative powers spanning this range are cumbersome to work with.
For instance, if I have a signal whose energy is 6 millwatts and amplify it to 5 megawatts,
the amplification is given by their ratio, the gain,
%
\begin{displaymath}
  \mathrm{Gain}\; = \frac{P_{out}}{P_{in}} = \frac{5\times 10^6}{6\times 10^{-3}}
  \approx 8.3\times 10^8 \,.
\end{displaymath}
%
The simplest way to overcome this is to invoke a logarithmic scale for relative
powers, resulting in a value on the Bel scale,
%
\begin{displaymath}
  \mathrm{Gain}\;\mathrm{(Bel)}\; = \log_{10}\left(\frac{P_{out}}{P_{in}}\right)
  = \log_{10}\left(\frac{5\times 10^6}{6\times 10^{-3}}\right) \approx 8.9\; \mathrm{Bel} \,.
\end{displaymath}
%
A quirk of history means that we typically work with decibels rather than Bels,
%
\begin{displaymath}
  \mathrm{Gain}\;\mathrm{(dB)} = 10\log_{10}\left(\frac{P_{out}}{P_{in}}\right)\; \mathrm{dB} \,,
\end{displaymath}
%
the reason being that the smallest detectable difference in volume by a human is
a decibel, which has led to its use in other fields by convention. There is another
scale which uses the natural logarithm, the Neper Scale, but this has never gained
traction. Inverting our expression for the gain to recover the power of the output signal,
%
\begin{displaymath}
  \log_{10}\left(\frac{P_{out}}{P_{in}}\right) = \frac{\mathrm{Gain}}{10}
  \quad\rightarrow\quad
  P_{out} = 10^{\frac{Gain}{10}}P_{in} \,.
\end{displaymath}
%
In signal processing, another common unit is the ``decibel milliwatt'', or dBm, where
the reference input power is equal to a milliwatt,
%
\begin{displaymath}
  \mathrm{Gain}\;\mathrm{(dBm)} = 10\log_{10}\left(\frac{P_{out}}{1\mathrm{mW}}\right)\; \mathrm{dBm} \,.
\end{displaymath}
%
A second useful feature of these logarithmic ratios is the fact that multiplication
of gains in the dimensionless scale we first introduced becomes addition.
%
\begin{exmp}
  Consider an initial signal with power $7$dBm, which is amplified by an amplifier with
  a gain of 2. In dBm,
  %
  \begin{displaymath}
    10\log_{10}\left(\frac{2\mathrm{mW}}{1\mathrm{mW}}\right) \approx 3\mathrm{dBm} \,.
  \end{displaymath}
  %
  The power of the resultant signal is simply given by the sum of the initial signal power
  and the gain of the amplifier, $7\mathrm{dBm}\; + 3\mathrm{dBm}\; = 10\mathrm{dBm}$.
\end{exmp}
%
It is worth considering the form of the gain in units of decibels -- it is plotted
in Figure \ref{fig::appendix_1_decibels}. Note that when the input and output signals
have the same power, the gain is equal to $1$dB. In keeping with the additive property
of decibels, any decrease in gain is given by a negative number.
%
\begin{figure}[!htb]
  \begin{tikzpicture}
    \begin{axis}[
        width=0.9\textwidth,
        axis lines=middle,
        xmin=0, xmax=10, ymin=-10, ymax=10,
        xlabel=$\frac{P_{out}}{P_{in}}$,
        ylabel=$\mathrm{Gain}\;\mathrm{(dB)}$
      ]
      \addplot[samples=100, domain=0:15, blue, ultra thick] {10*log10(x)};
    \end{axis}
  \end{tikzpicture}
  \caption{The ratio of output to input signals plotted against the associated
    gain in dB. Observe that a ratio of 2, i.e. the output signal has a power twice
    that of the input signal, the gain is 3dB.}
  \label{fig::appendix_1_decibels}
\end{figure}

