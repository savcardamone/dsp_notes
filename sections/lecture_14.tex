\section{Lecture 14: Upsampling and Downsampling}

\subsection{Discrete-Time Processing of Continuous-Time Signals}
%
In the previous lecture, we have a continuous-time signal, $x_c(t)$ which, when
multiplied by the impulse train $d(t)$ yields a second continuous-time signal,
$x_s(t)$, whose non-zero values are spaced by $T$ in time and is, in effect,
a continuous-time version of a discrete-time signal. $x_s(t)$ would then be
converted into a discrete-time signal $x_d[n]$. We also showed that for the CTFT,
%
\begin{displaymath}
  X_s(\omega) = \frac{1}{T}\infsum{k}X_c(\omega - k\omega_s) \,,
\end{displaymath}
%
where $\omega_s = \frac{2\pi}{T}$. The DTFT of $x_d[n]$ is retrieved by simply
recalling that it is $2\pi$-periodic, and so the copies that appear in
$X_s(\omega)$ are spaced $2\pi$ apart rather than $\omega_s$. The next piece of
the puzzle involves the processing of $x_d[n]$ by some discrete-time system,
resulting in the output discrete-time signal $y_d[n]$. Finally, we may wish to
convert this back to a continuous-time signal, $y_r(t)$. Our question then is
how do we create the discrete-time system such that it behaves like the
frequency response of a continuous-time system.\\

We know that the frequency responses of the first analog-to-digital stage is
%
\begin{displaymath}
  X_d(\omega) = \frac{1}{T}\infsum{k}X_c\left(\frac{\omega - 2\pi k}{T}\right) \,,
\end{displaymath}
%
and that of the final digital-to-analog state is
%
\begin{displaymath}
  Y_r(\omega) = H_r(\omega)Y_d(\omega T) = \infsum{n} y_d[n]\sinc\left(\frac{t - nT}{T}\right) \,.
\end{displaymath}
%
The middle stage can be modelled as
%
\begin{displaymath}
  Y_d(\omega) = H(\omega)X_d(\omega) \,,
\end{displaymath}
%
where $H(\omega)$ is the general frequency-response of the discrete-time system.
We can simply bolt all of these stages together since the output of one
stage forms the input of another,
%
\begin{displaymath}
  Y_r(\omega) = H_r(\omega)Y_d(\omega T) = H_r(\omega)H(\omega)X_d(\omega)
  = H_r(\omega)H(\omega T)\frac{1}{T}\infsum{k}X_c\left(\omega -\frac{2\pi k}{T}\right) \,.
\end{displaymath}
%
Now, if we enforce the band-limit, such that for $|\omega| > \frac{\pi}{T}, X_c(\omega) = 0$,
then
%
\begin{displaymath}
  Y_r(\omega) = \left\{\begin{array}{ccl}
  H(\omega T)X_c(\omega) & & |\omega| < \frac{\pi}{T} \\
  0 & & \mathrm{otherwise}
  \end{array}\right. \,,
\end{displaymath}
%
since $H_r(\omega)$ is a band-pass filter. As such, the effective continuous-time frequency
response is
%
\begin{displaymath}
  Y_r(\omega) = H_{\mathrm{eff}}(\omega)X(\omega) \,,
\end{displaymath}
%
where
%
\begin{displaymath}
  H_{\mathrm{eff}}(\omega) = \left\{\begin{array}{ccl}
  H(\omega T) & & |\omega| < \frac{\pi}{T} \\
  0 & & \mathrm{otherwise}
  \end{array}\right. \,.
\end{displaymath}
%
Consequently, we can only design filters which are active when $|\omega| < \frac{\pi}{T}$,
else the reconstruction filter will remove any effects outside of this range; if we
want to make this interval wider, then we need to change the sampling rate.
