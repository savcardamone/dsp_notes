\section{Lecture 15: Multirate Signal Processing and Polyphase Representations}
%
\subsection{Changing the Sample Rate By a Non-Integer Factor}
%
Previously, we discussed downsampling and upsampling by an integer rate.
For downsampling by a factor of $M$, we first prefiltered with a low-pass
filter with cutoff frequency $\pi/M$ and gain of $1$ to prevent aliasing,
then downsampled by $M$. For upsampling by a factor of $L$, we padded our
signal with $L-1$ zeroes between each of the original samples, then filtered
with a low-pass filter with cutoff frequency $\pi/L$ and gain of $L$
to interpolate.\\
%
Upsampling or downsampling by integer rates is rather restrictive; there are
many situations where we'd like to use non-integer rates. Consider changing
the sampling rate by the non-integer factor $\tau$, which can be written as
the quotient of two integers, $M/L$. We could in principle make some combination
of upsampling by $L$ and downsampling by $M$ to achieve this. So,
%
\begin{displaymath}
  x[n] \longrightarrow \boxed{\uparrow L}
  \longrightarrow \boxed{H_\mathrm{int}(\omega)}
  \longrightarrow \boxed{H_\mathrm{pre}(\omega)}
  \longrightarrow \boxed{\downarrow M}
  \longrightarrow x_\tau[n] \,.
\end{displaymath}
%
Note that the interpolating filter and prefilter can be condensed into a single
filter; in the frequency domain, we need simply multiply them together, which
results in a single low-pass filter whose cutoff frequency is dictated by
the smallest cutoff frequency of the other two,
%
\begin{displaymath}
  H_\tau(\omega) = H_\mathrm{int}(\omega)H_\mathrm{pre}(\omega)
  \quad \omega_c = \min\left(\frac{\pi}{M},\frac{\pi}{L}\right) \,.
\end{displaymath}
%
If $M>L$, we have a net reduction in sampling rate, meaning $H_\tau(\omega)$
is $H_\mathrm{pre}(\omega)$, and we need to prevent possible aliasing. If
$L>M$, we have a net increase in sampling rate, meaning $H_\tau(\omega)$ is
$H_\mathrm{int}(\omega)$, and we need to interpolate the signal. To give some
numbers, assume $\tau = 1.2$, then we need to upsample by a factor of $5$
and downsample by a factor of $6$, meaning we need to use
$H_\mathrm{pre}(\omega)$.

\subsection{The Noble Identities}
%
This strategy is problematic. Consider $\tau = 1.01$, where $M = 101$ and
$L = 100$, which seems rather wasteful since we're implementing large intermediate
changes in rate to retrieve a signal which is close to the original rate. We then
ask whether this process can be done more efficiently, to which the answer is of
course yes, and is referred to as \textbf{multirate signal processing}.
%
\begin{iden}[\textbf{The Noble Identity for Decimation}]
  The processes
  %
  \begin{displaymath}
    x[n] \longrightarrow \boxed{\downarrow M} \longrightarrow x_a[n]
    \longrightarrow \boxed{H(z)} \longrightarrow y_a[n]
  \end{displaymath}
  %
  and
  %
  \begin{displaymath}
    x[n] \longrightarrow \boxed{H\left(z^M\right)} \longrightarrow x_b[n]
    \longrightarrow \boxed{\downarrow M} \longrightarrow y_b[n]
  \end{displaymath}
  %
  are equivalent.\\

  Let's consider the meaning of $H(z^M)$ for a little insight:
  %
  \begin{displaymath}
    H(z^M) = \infsum{n} h[n]z^{-Mn} = \mathscr{Z}(h_e[n]) \,,
  \end{displaymath}
  %
  where $h_e[n]$ is a zero-padded version of $h[n]$, with $M-1$ zeroes between
  each sample. Recall that to transform between the $\omega$ and $z$ domains,
  we have the relationship
  %
  \begin{displaymath}
    X(\omega) = \left. X(z) \right|_{z = \ex{\im\omega}} \,.
  \end{displaymath}
  %
  Consequently, we can write the expression
  %
  \begin{displaymath}
    X_b(\omega) = H\left(z^M\right)X(\omega) = H\left(\ex{\im\omega M}\right)X(\omega)
    = H(\omega M) X(\omega)
  \end{displaymath}
  %
  since
  %
  \begin{displaymath}
    \left.H\left(z^M\right)\right|_{z=\ex{\im\omega}}
    = \infsum x[n]\ex{\im\omega nM} = H(\omega M) \,.
  \end{displaymath}
  %
  We saw in the previous lecture that the downsampling block of $X_b(\omega)$ results in
  %
  \begin{displaymath}
    Y_b(\omega) = \frac{1}{M}\sum_{m=0}^{M-1}X\left(\frac{\omega - 2\pi m}{M}\right)
    H(\omega - 2\pi m)
    = H(\omega)\frac{1}{M}\sum_{m=0}^{M-1} X\left(\frac{\omega - 2\pi m}{M}\right) \,,
  \end{displaymath}
  %
  since $H(\omega)$ is $2\pi$-periodic. But this is simply $H(\omega)$ multiplied by
  the result from having first downsampled by $X(\omega)$ by $M$, and consequently we
  have proven that $Y_b(\omega) = Y_a(\omega)$.
\end{iden}
%
\begin{iden}[\textbf{The Noble Identity for Interpolation}]
  The processes
  %
  \begin{displaymath}
    x[n] \longrightarrow \boxed{\uparrow L} \longrightarrow x_a[n]
    \longrightarrow \boxed{H(z)} \longrightarrow y_a[n]
  \end{displaymath}
  %
  and
  %
  \begin{displaymath}
    x[n] \longrightarrow \boxed{H\left(z^L\right)} \longrightarrow x_b[n]
    \longrightarrow \boxed{\uparrow L} \longrightarrow y_b[n]
  \end{displaymath}
  %
  are equivalent.\\

  Recall that upsampling shrinks the spectrum in the frequency domain by a factor
  $L$. Then
  %
  \begin{displaymath}
    Y_a(\omega) = X_a(\omega L) = X(\omega L)H(\omega L) \,,
  \end{displaymath}
  %
  and similarly
  %
  \begin{displaymath}
    X_b(\omega) = X(\omega L) \,,
  \end{displaymath}
  %
  meaning
  %
  \begin{displaymath}
    Y_b(\omega) = X(\omega L)H(\omega L) \,,
  \end{displaymath}
  %
  and we've proven that $Y_a(\omega) = Y_b(\omega)$. This seemingly innocent looking
  interchange has important computational consequences. If we upsample first, we increase
  our signal length by a factor of $L$, most of the elements being zero. This then
  results in a lot of redundant computation by the subsequent filter which has to
  process the signal in serial. In contrast, if we filter first, we have a significantly
  reduced computational overhead, which can later be upsampled.
\end{iden}

\subsection{The Polyphase Decomposition}
%
Consider some signal $h[n]$, an example of which is depicted in Figure QQ. We can think
of this sum as a sum of simpler signals; $h[n]$ is decomposed into a sum of $M$
subsequences $h_k[n]$, defined as follows,
%
\begin{displaymath}
  h_k[n] = h[k + nM] \,,
\end{displaymath}
%
which is the original sequence delayed by $k$ units and zero-padded with $M-1$ zeroes
between each of the originally sampled points. This is simplest to understand pictorially;
for $M=3$, pane (b) of Figure QQ shows the decomposition. We see that the original
signal is simply the summation of these subsequences,
%
\begin{displaymath}
  h[n] = \sum_{k=0}^{M-1}h_k[n-k] \,.
\end{displaymath}
%
Let $e_k[n]$ denote a corresponding $h_k[n]$ whose zero-padding has been removed, i.e.
$e_k[n] = h_k[nM]$. These are termed the polyphase components of $h[n]$.
